\documentclass[12pt]{book}
%\usepackage[papersize={125mm,205mm},tmargin=1.5cm,bmargin=1.5cm,hmargin=1cm]{geometry}
\usepackage[utf8x]{inputenc}
\usepackage{../preambule_NSI_2021/MichaelNSI2021}


\setcounter{minitocdepth}{1}
\usepackage{qrcode}

\mtcselectlanguage{francais}


%%%%%%%% Faire de l'air autour d'une fraction (surtout ds un tableau)
\newcommand{\delair}[1]{\ensuremath\displaystyle\psframebox[framesep=0.15em,linestyle=none]{ \displaystyle1}}


\fancyhead[L]{\includegraphics[scale=0.03]{../preambule_NSI_2021/LATEXimages/NSI_600}\normalsize N.S.I : Numérique et Science Informatique}
%\fancyhead[R]{\normalsize \rightmark}
\renewcommand\headrulewidth{1pt}
\renewcommand\footrulewidth{1pt}
%\fancyfoot[L]{Terminale $S$}
\fancyfoot[C]{\includegraphics[scale=0.75]{../preambule_NSI_2021/creative_licence_compact}}%\textbf{Page \thepage/\pageref{LastPage}}}
%\fancyfoot[R]{\textbf{Page \thepage}}%\pageref{LastPage}}}


\fancyfoot[LE]{$\blacksquare$\,$\blacksquare$\;\thechapter.\thepage}%
\fancyfoot[RO]{\thechapter.\thepage\;$\blacksquare$\,$\blacksquare$}%
\fancyfoot[LO]{\uppercase{\sffamily{Chapitre \thechapter}}}
\fancyfoot[RE]{\normalsize\sffamily Chapitre \thechapter }
\pagestyle{fancy}

\renewcommand{\arraystretch}{1} 
\renewcommand*\tabularxcolumn[1]{>{\centering\arraybackslash}m{#1}}
%\usepackage[bloc,completemulti]{automultiplechoice}  

\usepackage{comment}
%\affichecorrectionfalse
%\affichecorrectiontrue
%\affichepreuvefalse
%\excludecomment{preuve}



\usepackage{tcolorbox}

\usepackage{arydshln}

\usepackage{auto-pst-pdf}
%\usepackage{pdftricks}
%\begin{psinputs}
%	\usepackage{pstricks,pst-plot,pst-text,pst-tree,pst-eps,pst-fill,pst-node,pst-math,pst-eucl,pst-all}
%	\usepackage{multido}
%\end{psinputs}

\usepackage{picins}
\usepackage[tikz]{bclogo}
\author{M.Meyroneinc}
\title{Première}


%---- Mise en diapo :
%\usepackage[
%screen,
%%print,
%%panelright,
%nopanel,
%%paneltoc,
%bluelace,
%french,
%sectionbreak]{
%	pdfscreen}
%\margins{.75in}{.75in}{.75in}{.75in}
%\screensize{6.25in}{8in}
%\overlay{overlay4.pdf}
%%\overlay{lightsteelblue.pdf }
%\panelhomepagename{Page d’accueil}
%\panelfullscreenname{Plein écran}
%\bottombuttons
%\pagedissolve{/Box}


\def\colorminted{gray!10!white}

% Ajustement de l'espace dans la table des matières
\newlength{\ecartnumero}

\setlength{\ecartnumero}{4mm}
%\usepackage{titlesec}
\usepackage{titletoc}
\titlecontents{chapter}%
[\dimexpr 2.5mm+\ecartnumero]
{\vspace{2.2mm}\bfseries}
{\contentslabel{\dimexpr 10mm+\ecartnumero}}
{\hspace{\dimexpr -5.5mm-\ecartnumero}}
{\hfill\contentspage}

\dottedcontents{section}%
[\dimexpr 15mm+\ecartnumero]
{}
{\dimexpr 15mm+\ecartnumero}
{3.2mm}

\dottedcontents{subsection}%
[\dimexpr 20mm+\ecartnumero]
{}
{\dimexpr 15mm+\ecartnumero}
{3.2mm}

\dottedcontents{subsubsection}%
[\dimexpr 25mm+\ecartnumero]
{}
{\dimexpr 15mm+\ecartnumero}
{3.2mm}

%\usepackage{marvosym}

\begin{document}
   \dominitoc
  
\theme{T6}
\Tsix
\faketableofcontents
\setcounter{chapter}{22}

\chapter{Les Dictionnaires}
\minitoc     
 


\begin{center}\rule{0.5\linewidth}{0.5pt}\end{center}

    \hypertarget{introduction}{%
\section{Introduction}\label{introduction}}

Prenons l'exemple d'un répertoire téléhonique. Nous pouvons le mémoriser
simplement comme un tableau (ou liste) de tableaux
\texttt{{[}nom,numéro{]}}

    \begin{tcolorbox}[breakable, size=fbox, boxrule=1pt, pad at break*=1mm,colback=cellbackground, colframe=cellborder]
\prompt{In}{incolor}{ }{\boxspacing}
\begin{Verbatim}[commandchars=\\\{\}]
\PY{n}{liste\PYZus{}tel} \PY{o}{=} \PY{p}{[}\PY{p}{[}\PY{l+s+s2}{\PYZdq{}}\PY{l+s+s2}{Paul}\PY{l+s+s2}{\PYZdq{}}\PY{p}{,} \PY{l+m+mi}{5234}\PY{p}{]}\PY{p}{,}
             \PY{p}{[}\PY{l+s+s2}{\PYZdq{}}\PY{l+s+s2}{Emile}\PY{l+s+s2}{\PYZdq{}}\PY{p}{,} \PY{l+m+mi}{5345}\PY{p}{]}\PY{p}{,}
             \PY{p}{[}\PY{l+s+s2}{\PYZdq{}}\PY{l+s+s2}{Victor}\PY{l+s+s2}{\PYZdq{}}\PY{p}{,} \PY{l+m+mi}{5186}\PY{p}{]}\PY{p}{,}
             \PY{p}{[}\PY{l+s+s2}{\PYZdq{}}\PY{l+s+s2}{Rose}\PY{l+s+s2}{\PYZdq{}}\PY{p}{,} \PY{l+m+mi}{5678}\PY{p}{]}\PY{p}{,}
             \PY{p}{[}\PY{l+s+s2}{\PYZdq{}}\PY{l+s+s2}{Hélène}\PY{l+s+s2}{\PYZdq{}}\PY{p}{,} \PY{l+m+mi}{5432}\PY{p}{]}\PY{p}{]}
\end{Verbatim}
\end{tcolorbox}

    Si nous voulons appeler \emph{Rose}, nous avons deux possibilités avec
un tel tableau : * soit il faut savoir que les informations la
concernant sont dans le quatrième élément de la liste (ce qui ne semble
pas très pratique et réaliste)

    \begin{tcolorbox}[breakable, size=fbox, boxrule=1pt, pad at break*=1mm,colback=cellbackground, colframe=cellborder]
\prompt{In}{incolor}{ }{\boxspacing}
\begin{Verbatim}[commandchars=\\\{\}]
\PY{n+nb}{print}\PY{p}{(}\PY{n}{liste\PYZus{}tel}\PY{p}{[}\PY{l+m+mi}{3}\PY{p}{]}\PY{p}{[}\PY{l+m+mi}{1}\PY{p}{]}\PY{p}{)} \PY{c+c1}{\PYZsh{} il faut savoir que l\PYZsq{}index de Rose est 3}
\end{Verbatim}
\end{tcolorbox}

    \begin{itemize}
\tightlist
\item
  soit nous cherchons dans le tableau en partant du premier élément de
  la liste jusqu'à ce que nous trouvions \emph{Rose} (ce qui revient à
  feuilleter son répertoire) : cela nécessite d'utiliser une boucle pour
  parcourir le tableau.
\end{itemize}

    \begin{tcolorbox}[breakable, size=fbox, boxrule=1pt, pad at break*=1mm,colback=cellbackground, colframe=cellborder]
\prompt{In}{incolor}{ }{\boxspacing}
\begin{Verbatim}[commandchars=\\\{\}]
\PY{k}{for} \PY{n}{element} \PY{o+ow}{in} \PY{n}{liste\PYZus{}tel}\PY{p}{:}
    \PY{k}{if} \PY{n}{element}\PY{p}{[}\PY{l+m+mi}{0}\PY{p}{]} \PY{o}{==} \PY{l+s+s1}{\PYZsq{}}\PY{l+s+s1}{Rose}\PY{l+s+s1}{\PYZsq{}}\PY{p}{:}
        \PY{n+nb}{print}\PY{p}{(}\PY{n}{element}\PY{p}{[}\PY{l+m+mi}{1}\PY{p}{]}\PY{p}{)}
\end{Verbatim}
\end{tcolorbox}

    \begin{Verbatim}[commandchars=\\\{\}]
5678
    \end{Verbatim}

    Vous conviendrez que ce n'est pas pratique pour accéder à son numéro de
téléphone. De même, la modification ou l'ajout d'un information
nécessiterait de devoir feuilleter tout le répertoire. Il semblerait
plus pratique d'associer un nom à un numéro, autrement dit d'associer à
une \textbf{information} à une \textbf{clé}.

C'est ce que les dictionnaires permettent !

    \hypertarget{les-dictionnaires-en-python}{%
\section{Les dictionnaires en
Python}\label{les-dictionnaires-en-python}}

    \begin{verbatim}
Un dictionnaire est un ensemble <strong>non ordonné</strong> de paires (clé, valeur) avec un accès très rapide à la valeur à partir de la clé.
\end{verbatim}

    On peut ajouter des couples (clé, valeur) à un dictionnaire, si la clé
figure déjà dans le dictionnaire alors le couple est remplacé par le
nouveau. Une \textbf{clé} peut être de type alphabétique, numérique, ou
même de type construit sous certaines conditions. Les \textbf{valeurs}
pourront être de tout type sans exclusion. En Python, le dictionnaire
est un objet \textbf{mutable}, autrement dit, on peut le modifier.

    \hypertarget{cruxe9ation-dun-dictionnaire}{%
\subsection{Création d'un
dictionnaire}\label{cruxe9ation-dun-dictionnaire}}

Plusieurs méthodes permettent de créer soit un dictionnaire vide, soit
de le noter en extension, soit par compréhension.

    \begin{tcolorbox}[breakable, size=fbox, boxrule=1pt, pad at break*=1mm,colback=cellbackground, colframe=cellborder]
\prompt{In}{incolor}{ }{\boxspacing}
\begin{Verbatim}[commandchars=\\\{\}]
\PY{n}{d1} \PY{o}{=} \PY{p}{\PYZob{}}\PY{p}{\PYZcb{}}     \PY{c+c1}{\PYZsh{} Création d\PYZsq{}un dictionnaire vide}
\PY{n}{d2} \PY{o}{=} \PY{n+nb}{dict}\PY{p}{(}\PY{p}{)} \PY{c+c1}{\PYZsh{} Création d\PYZsq{}un dictionnaire vide (autre méthode)}
\PY{n}{d3} \PY{o}{=} \PY{p}{\PYZob{}}\PY{l+s+s1}{\PYZsq{}}\PY{l+s+s1}{poires}\PY{l+s+s1}{\PYZsq{}}\PY{p}{:} \PY{l+m+mi}{5}\PY{p}{,} \PY{l+s+s1}{\PYZsq{}}\PY{l+s+s1}{bananes}\PY{l+s+s1}{\PYZsq{}}\PY{p}{:} \PY{l+m+mi}{7}\PY{p}{,} \PY{l+s+s1}{\PYZsq{}}\PY{l+s+s1}{abricots}\PY{l+s+s1}{\PYZsq{}} \PY{p}{:} \PY{l+m+mi}{12}\PY{p}{\PYZcb{}} \PY{c+c1}{\PYZsh{} création d\PYZsq{}un dictionnaire par extension}
\PY{n}{d4} \PY{o}{=} \PY{p}{\PYZob{}}\PY{n}{k}\PY{p}{:} \PY{n}{k}\PY{o}{*}\PY{o}{*}\PY{l+m+mi}{2} \PY{k}{for} \PY{n}{k} \PY{o+ow}{in} \PY{n+nb}{range}\PY{p}{(}\PY{l+m+mi}{1}\PY{p}{,} \PY{l+m+mi}{10}\PY{p}{)}\PY{p}{\PYZcb{}} \PY{c+c1}{\PYZsh{} création d\PYZsq{}un dictionnaire par compréhension}

\PY{n+nb}{print}\PY{p}{(}\PY{n+nb}{type}\PY{p}{(}\PY{n}{d1}\PY{p}{)}\PY{p}{)}
\end{Verbatim}
\end{tcolorbox}

    \begin{Verbatim}[commandchars=\\\{\}]
<class 'dict'>
    \end{Verbatim}

    \begin{tcolorbox}[breakable, size=fbox, boxrule=1pt, pad at break*=1mm,colback=cellbackground, colframe=cellborder]
\prompt{In}{incolor}{ }{\boxspacing}
\begin{Verbatim}[commandchars=\\\{\}]
\PY{n+nb}{print}\PY{p}{(}\PY{l+s+s2}{\PYZdq{}}\PY{l+s+s2}{d1 =\PYZgt{}}\PY{l+s+s2}{\PYZdq{}}\PY{p}{,} \PY{n}{d1}\PY{p}{)}
\PY{n+nb}{print}\PY{p}{(}\PY{l+s+s2}{\PYZdq{}}\PY{l+s+s2}{d2 =\PYZgt{}}\PY{l+s+s2}{\PYZdq{}}\PY{p}{,} \PY{n}{d2}\PY{p}{)}
\PY{n+nb}{print}\PY{p}{(}\PY{l+s+s2}{\PYZdq{}}\PY{l+s+s2}{d3 =\PYZgt{}}\PY{l+s+s2}{\PYZdq{}}\PY{p}{,} \PY{n}{d3}\PY{p}{)}
\PY{n+nb}{print}\PY{p}{(}\PY{l+s+s2}{\PYZdq{}}\PY{l+s+s2}{d4 =\PYZgt{}}\PY{l+s+s2}{\PYZdq{}}\PY{p}{,} \PY{n}{d4}\PY{p}{)}
\end{Verbatim}
\end{tcolorbox}

    \begin{Verbatim}[commandchars=\\\{\}]
d1 => \{\}
d2 => \{\}
d3 => \{'poires': 5, 'bananes': 7, 'abricots': 12\}
d4 => \{1: 1, 2: 4, 3: 9, 4: 16, 5: 25, 6: 36, 7: 49, 8: 64, 9: 81\}
    \end{Verbatim}

    Il est même possible de créer un dictionnaire à partir d'une liste de
couples.

    \begin{tcolorbox}[breakable, size=fbox, boxrule=1pt, pad at break*=1mm,colback=cellbackground, colframe=cellborder]
\prompt{In}{incolor}{ }{\boxspacing}
\begin{Verbatim}[commandchars=\\\{\}]
\PY{n}{liste} \PY{o}{=} \PY{p}{[}\PY{p}{(}\PY{l+s+s1}{\PYZsq{}}\PY{l+s+s1}{cle1}\PY{l+s+s1}{\PYZsq{}}\PY{p}{,}\PY{l+s+s1}{\PYZsq{}}\PY{l+s+s1}{valeur1}\PY{l+s+s1}{\PYZsq{}}\PY{p}{)}\PY{p}{,}\PY{p}{(}\PY{l+s+s1}{\PYZsq{}}\PY{l+s+s1}{cle2}\PY{l+s+s1}{\PYZsq{}}\PY{p}{,}\PY{l+s+s1}{\PYZsq{}}\PY{l+s+s1}{valeur2}\PY{l+s+s1}{\PYZsq{}}\PY{p}{)}\PY{p}{]}
\PY{n}{d5} \PY{o}{=} \PY{n+nb}{dict}\PY{p}{(}\PY{n}{liste}\PY{p}{)}
\PY{n}{liste\PYZus{}tel} \PY{o}{=} \PY{p}{[}\PY{p}{[}\PY{l+s+s2}{\PYZdq{}}\PY{l+s+s2}{Paul}\PY{l+s+s2}{\PYZdq{}}\PY{p}{,} \PY{l+m+mi}{5234}\PY{p}{]}\PY{p}{,} \PY{p}{[}\PY{l+s+s2}{\PYZdq{}}\PY{l+s+s2}{Emile}\PY{l+s+s2}{\PYZdq{}}\PY{p}{,} \PY{l+m+mi}{5345}\PY{p}{]}\PY{p}{,} \PY{p}{[}\PY{l+s+s2}{\PYZdq{}}\PY{l+s+s2}{Victor}\PY{l+s+s2}{\PYZdq{}}\PY{p}{,} \PY{l+m+mi}{5186}\PY{p}{]}\PY{p}{,} \PY{p}{[}\PY{l+s+s2}{\PYZdq{}}\PY{l+s+s2}{Rose}\PY{l+s+s2}{\PYZdq{}}\PY{p}{,} \PY{l+m+mi}{5678}\PY{p}{]}\PY{p}{,} \PY{p}{[}\PY{l+s+s2}{\PYZdq{}}\PY{l+s+s2}{Hélène}\PY{l+s+s2}{\PYZdq{}}\PY{p}{,} \PY{l+m+mi}{5432}\PY{p}{]}\PY{p}{]}
\PY{n}{d6} \PY{o}{=} \PY{n+nb}{dict}\PY{p}{(}\PY{n}{liste\PYZus{}tel}\PY{p}{)}

\PY{n+nb}{print}\PY{p}{(}\PY{l+s+s2}{\PYZdq{}}\PY{l+s+s2}{d5 =\PYZgt{}}\PY{l+s+s2}{\PYZdq{}}\PY{p}{,} \PY{n}{d5}\PY{p}{)}
\PY{n+nb}{print}\PY{p}{(}\PY{l+s+s2}{\PYZdq{}}\PY{l+s+s2}{d6 =\PYZgt{}}\PY{l+s+s2}{\PYZdq{}}\PY{p}{,} \PY{n}{d6}\PY{p}{)}
\end{Verbatim}
\end{tcolorbox}

    \begin{Verbatim}[commandchars=\\\{\}]
d5 => \{'cle1': 'valeur1', 'cle2': 'valeur2'\}
d6 => \{'Paul': 5234, 'Emile': 5345, 'Victor': 5186, 'Rose': 5678, 'Hélène':
5432\}
    \end{Verbatim}

    \begin{quote}
\textbf{Important} : Vous aurez noté que les dictionnaires Python se
représentent entre accolades \texttt{\{\}}. Les différentes paires sont
séparées par des virgules et sont de la forme \texttt{clé:\ valeur}.
\end{quote}

    \begin{center}\rule{0.5\linewidth}{0.5pt}\end{center}

\hypertarget{activituxe9-23.1}{%
\subsubsection{Activité 23.1}\label{activituxe9-23.1}}

Créez un dictionnaire appelé \texttt{notes} qui contient les paires
(matières, moyenne) de vos trois spécialités. Affichez ensuite ce
dictionnaire.

    \begin{tcolorbox}[breakable, size=fbox, boxrule=1pt, pad at break*=1mm,colback=cellbackground, colframe=cellborder]
\prompt{In}{incolor}{ }{\boxspacing}
\begin{Verbatim}[commandchars=\\\{\}]
\PY{c+c1}{\PYZsh{} à vous de jouer !}
\end{Verbatim}
\end{tcolorbox}

    \begin{center}\rule{0.5\linewidth}{0.5pt}\end{center}

\hypertarget{accuxe8s-modification-ajout-suppression}{%
\subsection{Accès, modification, ajout,
suppression}\label{accuxe8s-modification-ajout-suppression}}

L'\textbf{accès} à une valeur d'un dictionnaire se fait par sa clé.

    \begin{tcolorbox}[breakable, size=fbox, boxrule=1pt, pad at break*=1mm,colback=cellbackground, colframe=cellborder]
\prompt{In}{incolor}{ }{\boxspacing}
\begin{Verbatim}[commandchars=\\\{\}]
\PY{n}{d3} \PY{o}{=} \PY{p}{\PYZob{}}\PY{l+s+s1}{\PYZsq{}}\PY{l+s+s1}{poires}\PY{l+s+s1}{\PYZsq{}}\PY{p}{:} \PY{l+m+mi}{5}\PY{p}{,} \PY{l+s+s1}{\PYZsq{}}\PY{l+s+s1}{bananes}\PY{l+s+s1}{\PYZsq{}}\PY{p}{:} \PY{l+m+mi}{7}\PY{p}{,} \PY{l+s+s1}{\PYZsq{}}\PY{l+s+s1}{abricots}\PY{l+s+s1}{\PYZsq{}} \PY{p}{:} \PY{l+m+mi}{12}\PY{p}{\PYZcb{}}
\PY{n}{d3}\PY{p}{[}\PY{l+s+s1}{\PYZsq{}}\PY{l+s+s1}{abricots}\PY{l+s+s1}{\PYZsq{}}\PY{p}{]}
\end{Verbatim}
\end{tcolorbox}

            \begin{tcolorbox}[breakable, size=fbox, boxrule=.5pt, pad at break*=1mm, opacityfill=0]
\prompt{Out}{outcolor}{ }{\boxspacing}
\begin{Verbatim}[commandchars=\\\{\}]
12
\end{Verbatim}
\end{tcolorbox}
        
    Le dictionnaire étant un objet \emph{mutable} on peut \textbf{modifier}
la valeur associée à une clé ou \textbf{ajouter} une nouvelle
association et afficher le dictionnaire modifié.

    \begin{tcolorbox}[breakable, size=fbox, boxrule=1pt, pad at break*=1mm,colback=cellbackground, colframe=cellborder]
\prompt{In}{incolor}{ }{\boxspacing}
\begin{Verbatim}[commandchars=\\\{\}]
\PY{n}{d} \PY{o}{=} \PY{p}{\PYZob{}}\PY{l+s+s1}{\PYZsq{}}\PY{l+s+s1}{Paul}\PY{l+s+s1}{\PYZsq{}}\PY{p}{:} \PY{l+m+mi}{5234}\PY{p}{,} \PY{l+s+s1}{\PYZsq{}}\PY{l+s+s1}{Emile}\PY{l+s+s1}{\PYZsq{}}\PY{p}{:} \PY{l+m+mi}{5345}\PY{p}{,} \PY{l+s+s1}{\PYZsq{}}\PY{l+s+s1}{Victor}\PY{l+s+s1}{\PYZsq{}}\PY{p}{:} \PY{l+m+mi}{5186}\PY{p}{,} \PY{l+s+s1}{\PYZsq{}}\PY{l+s+s1}{Rose}\PY{l+s+s1}{\PYZsq{}}\PY{p}{:} \PY{l+m+mi}{5678}\PY{p}{,} \PY{l+s+s1}{\PYZsq{}}\PY{l+s+s1}{Hélène}\PY{l+s+s1}{\PYZsq{}}\PY{p}{:} \PY{l+m+mi}{5432}\PY{p}{\PYZcb{}}
\PY{n}{d}\PY{p}{[}\PY{l+s+s1}{\PYZsq{}}\PY{l+s+s1}{Rose}\PY{l+s+s1}{\PYZsq{}}\PY{p}{]} \PY{o}{=} \PY{l+m+mi}{4921}    \PY{c+c1}{\PYZsh{} clé existante donc modification de la valeur}
\PY{n}{d}\PY{p}{[}\PY{l+s+s1}{\PYZsq{}}\PY{l+s+s1}{Louane}\PY{l+s+s1}{\PYZsq{}}\PY{p}{]} \PY{o}{=} \PY{l+m+mi}{4118}  \PY{c+c1}{\PYZsh{} nouvelle clé donc ajout d\PYZsq{}une nouvelle association}
\PY{n+nb}{print}\PY{p}{(}\PY{n}{d}\PY{p}{)}
\end{Verbatim}
\end{tcolorbox}

    \begin{Verbatim}[commandchars=\\\{\}]
\{'Paul': 5234, 'Emile': 5345, 'Victor': 5186, 'Rose': 4921, 'Hélène': 5432,
'Louane': 4118\}
    \end{Verbatim}

    Pour \textbf{supprimer} une association d'un dictionnaire on peut
utilise le mot clé \texttt{del}.

    \begin{tcolorbox}[breakable, size=fbox, boxrule=1pt, pad at break*=1mm,colback=cellbackground, colframe=cellborder]
\prompt{In}{incolor}{ }{\boxspacing}
\begin{Verbatim}[commandchars=\\\{\}]
\PY{n+nb}{print}\PY{p}{(}\PY{n}{d}\PY{p}{)}
\PY{k}{del} \PY{n}{d}\PY{p}{[}\PY{l+s+s1}{\PYZsq{}}\PY{l+s+s1}{Paul}\PY{l+s+s1}{\PYZsq{}}\PY{p}{]}
\PY{n+nb}{print}\PY{p}{(}\PY{n}{d}\PY{p}{)}
\end{Verbatim}
\end{tcolorbox}

    \begin{Verbatim}[commandchars=\\\{\}]
\{'Paul': 5234, 'Emile': 5345, 'Victor': 5186, 'Rose': 4921, 'Hélène': 5432,
'Louane': 4118\}
\{'Emile': 5345, 'Victor': 5186, 'Rose': 4921, 'Hélène': 5432, 'Louane': 4118\}
    \end{Verbatim}

    \hypertarget{taille-dun-dictionnaire}{%
\subsection{Taille d'un dictionnaire}\label{taille-dun-dictionnaire}}

La fonction \texttt{len} renvoie la taille d'un dictionnaire.

    \begin{tcolorbox}[breakable, size=fbox, boxrule=1pt, pad at break*=1mm,colback=cellbackground, colframe=cellborder]
\prompt{In}{incolor}{ }{\boxspacing}
\begin{Verbatim}[commandchars=\\\{\}]
\PY{n}{d3} \PY{o}{=} \PY{p}{\PYZob{}}\PY{l+s+s1}{\PYZsq{}}\PY{l+s+s1}{poires}\PY{l+s+s1}{\PYZsq{}}\PY{p}{:} \PY{l+m+mi}{5}\PY{p}{,} \PY{l+s+s1}{\PYZsq{}}\PY{l+s+s1}{bananes}\PY{l+s+s1}{\PYZsq{}}\PY{p}{:} \PY{l+m+mi}{7}\PY{p}{,} \PY{l+s+s1}{\PYZsq{}}\PY{l+s+s1}{abricots}\PY{l+s+s1}{\PYZsq{}} \PY{p}{:} \PY{l+m+mi}{12}\PY{p}{\PYZcb{}}
\PY{n+nb}{len}\PY{p}{(}\PY{n}{d3}\PY{p}{)}
\end{Verbatim}
\end{tcolorbox}

            \begin{tcolorbox}[breakable, size=fbox, boxrule=.5pt, pad at break*=1mm, opacityfill=0]
\prompt{Out}{outcolor}{ }{\boxspacing}
\begin{Verbatim}[commandchars=\\\{\}]
3
\end{Verbatim}
\end{tcolorbox}
        
    \begin{center}\rule{0.5\linewidth}{0.5pt}\end{center}

\hypertarget{activituxe9-23.2}{%
\subsubsection{Activité 23.2}\label{activituxe9-23.2}}

On reprend le dictionnaire \texttt{notes} de l'activité 1.

\begin{enumerate}
\def\labelenumi{\arabic{enumi}.}
\tightlist
\item
  Affichez la moyenne de NSI.
\end{enumerate}

    \begin{tcolorbox}[breakable, size=fbox, boxrule=1pt, pad at break*=1mm,colback=cellbackground, colframe=cellborder]
\prompt{In}{incolor}{ }{\boxspacing}
\begin{Verbatim}[commandchars=\\\{\}]

\end{Verbatim}
\end{tcolorbox}

    \begin{enumerate}
\def\labelenumi{\arabic{enumi}.}
\setcounter{enumi}{1}
\tightlist
\item
  Modifiez votre moyenne de NSI qui a gagné 2 points. Affichez le
  dictionnaire.
\end{enumerate}

    \begin{tcolorbox}[breakable, size=fbox, boxrule=1pt, pad at break*=1mm,colback=cellbackground, colframe=cellborder]
\prompt{In}{incolor}{ }{\boxspacing}
\begin{Verbatim}[commandchars=\\\{\}]

\end{Verbatim}
\end{tcolorbox}

    \begin{enumerate}
\def\labelenumi{\arabic{enumi}.}
\setcounter{enumi}{2}
\tightlist
\item
  Ajoutez la matière \texttt{Anglais} avec sa moyenne. Affichez le
  dictionnaire.
\end{enumerate}

    \begin{tcolorbox}[breakable, size=fbox, boxrule=1pt, pad at break*=1mm,colback=cellbackground, colframe=cellborder]
\prompt{In}{incolor}{ }{\boxspacing}
\begin{Verbatim}[commandchars=\\\{\}]

\end{Verbatim}
\end{tcolorbox}

    \begin{enumerate}
\def\labelenumi{\arabic{enumi}.}
\setcounter{enumi}{3}
\tightlist
\item
  Affichez la taille du dictionnaire.
\end{enumerate}

    \begin{tcolorbox}[breakable, size=fbox, boxrule=1pt, pad at break*=1mm,colback=cellbackground, colframe=cellborder]
\prompt{In}{incolor}{ }{\boxspacing}
\begin{Verbatim}[commandchars=\\\{\}]

\end{Verbatim}
\end{tcolorbox}

    \begin{enumerate}
\def\labelenumi{\arabic{enumi}.}
\setcounter{enumi}{4}
\tightlist
\item
  Supprimez une des trois spécialités et affichez le dictionnaire.
\end{enumerate}

    \begin{tcolorbox}[breakable, size=fbox, boxrule=1pt, pad at break*=1mm,colback=cellbackground, colframe=cellborder]
\prompt{In}{incolor}{ }{\boxspacing}
\begin{Verbatim}[commandchars=\\\{\}]

\end{Verbatim}
\end{tcolorbox}

    \begin{center}\rule{0.5\linewidth}{0.5pt}\end{center}

\hypertarget{les-ituxe9rateurs-pour-les-dictionnaires}{%
\section{Les itérateurs pour les
dictionnaires}\label{les-ituxe9rateurs-pour-les-dictionnaires}}

Il est possible de parcourir un dictionnaire de trois manières :

\begin{itemize}
\tightlist
\item
  parcourir l'ensemble des \textbf{clés} avec la méthode \texttt{keys()}
  ;
\item
  parcourir l'ensemble des \textbf{valeurs} avec la méthode
  \texttt{values()} ;
\item
  parcourir l'ensemble des \textbf{paires clés-valeurs} avec la méthode
  \texttt{items()}.
\end{itemize}

On peut itérer sur un dictionnaire grâce à l'une de ces méthodes.

    \begin{tcolorbox}[breakable, size=fbox, boxrule=1pt, pad at break*=1mm,colback=cellbackground, colframe=cellborder]
\prompt{In}{incolor}{ }{\boxspacing}
\begin{Verbatim}[commandchars=\\\{\}]
\PY{n}{d} \PY{o}{=} \PY{p}{\PYZob{}}\PY{l+s+s1}{\PYZsq{}}\PY{l+s+s1}{Paul}\PY{l+s+s1}{\PYZsq{}}\PY{p}{:} \PY{l+m+mi}{5234}\PY{p}{,} \PY{l+s+s1}{\PYZsq{}}\PY{l+s+s1}{Emile}\PY{l+s+s1}{\PYZsq{}}\PY{p}{:} \PY{l+m+mi}{5345}\PY{p}{,} \PY{l+s+s1}{\PYZsq{}}\PY{l+s+s1}{Victor}\PY{l+s+s1}{\PYZsq{}}\PY{p}{:} \PY{l+m+mi}{5186}\PY{p}{,} \PY{l+s+s1}{\PYZsq{}}\PY{l+s+s1}{Rose}\PY{l+s+s1}{\PYZsq{}}\PY{p}{:} \PY{l+m+mi}{5678}\PY{p}{,} \PY{l+s+s1}{\PYZsq{}}\PY{l+s+s1}{Hélène}\PY{l+s+s1}{\PYZsq{}}\PY{p}{:} \PY{l+m+mi}{5432}\PY{p}{\PYZcb{}}
\PY{k}{for} \PY{n}{prenom} \PY{o+ow}{in} \PY{n}{d}\PY{o}{.}\PY{n}{keys}\PY{p}{(}\PY{p}{)}\PY{p}{:}
    \PY{n+nb}{print}\PY{p}{(}\PY{n}{prenom}\PY{p}{)}
\end{Verbatim}
\end{tcolorbox}

    \begin{Verbatim}[commandchars=\\\{\}]
Paul
Emile
Victor
Rose
Hélène
    \end{Verbatim}

    \begin{tcolorbox}[breakable, size=fbox, boxrule=1pt, pad at break*=1mm,colback=cellbackground, colframe=cellborder]
\prompt{In}{incolor}{ }{\boxspacing}
\begin{Verbatim}[commandchars=\\\{\}]
\PY{k}{for} \PY{n}{num} \PY{o+ow}{in} \PY{n}{d}\PY{o}{.}\PY{n}{values}\PY{p}{(}\PY{p}{)}\PY{p}{:}
    \PY{n+nb}{print}\PY{p}{(}\PY{n}{num}\PY{p}{)}
\end{Verbatim}
\end{tcolorbox}

    \begin{Verbatim}[commandchars=\\\{\}]
5234
5345
5186
5678
5432
    \end{Verbatim}

    \begin{tcolorbox}[breakable, size=fbox, boxrule=1pt, pad at break*=1mm,colback=cellbackground, colframe=cellborder]
\prompt{In}{incolor}{ }{\boxspacing}
\begin{Verbatim}[commandchars=\\\{\}]
\PY{k}{for} \PY{n}{prenom}\PY{p}{,} \PY{n}{num} \PY{o+ow}{in} \PY{n}{d}\PY{o}{.}\PY{n}{items}\PY{p}{(}\PY{p}{)}\PY{p}{:}
    \PY{n+nb}{print}\PY{p}{(}\PY{n}{prenom}\PY{p}{,} \PY{l+s+s1}{\PYZsq{}}\PY{l+s+s1}{\PYZhy{}\PYZgt{}}\PY{l+s+s1}{\PYZsq{}}\PY{p}{,} \PY{n}{num}\PY{p}{)}
\end{Verbatim}
\end{tcolorbox}

    \begin{Verbatim}[commandchars=\\\{\}]
Paul -> 5234
Emile -> 5345
Victor -> 5186
Rose -> 5678
Hélène -> 5432
    \end{Verbatim}

    On peut aussi interroger l'appartenance d'une valeur ou d'une clé grâce
au mot clé \texttt{in}.

    \begin{tcolorbox}[breakable, size=fbox, boxrule=1pt, pad at break*=1mm,colback=cellbackground, colframe=cellborder]
\prompt{In}{incolor}{ }{\boxspacing}
\begin{Verbatim}[commandchars=\\\{\}]
\PY{l+s+s1}{\PYZsq{}}\PY{l+s+s1}{John}\PY{l+s+s1}{\PYZsq{}} \PY{o+ow}{in} \PY{n}{d}\PY{o}{.}\PY{n}{keys}\PY{p}{(}\PY{p}{)}
\end{Verbatim}
\end{tcolorbox}

            \begin{tcolorbox}[breakable, size=fbox, boxrule=.5pt, pad at break*=1mm, opacityfill=0]
\prompt{Out}{outcolor}{ }{\boxspacing}
\begin{Verbatim}[commandchars=\\\{\}]
False
\end{Verbatim}
\end{tcolorbox}
        
    \begin{tcolorbox}[breakable, size=fbox, boxrule=1pt, pad at break*=1mm,colback=cellbackground, colframe=cellborder]
\prompt{In}{incolor}{ }{\boxspacing}
\begin{Verbatim}[commandchars=\\\{\}]
\PY{l+s+s1}{\PYZsq{}}\PY{l+s+s1}{Paul}\PY{l+s+s1}{\PYZsq{}} \PY{o+ow}{not} \PY{o+ow}{in} \PY{n}{d}\PY{o}{.}\PY{n}{keys}\PY{p}{(}\PY{p}{)}
\end{Verbatim}
\end{tcolorbox}

            \begin{tcolorbox}[breakable, size=fbox, boxrule=.5pt, pad at break*=1mm, opacityfill=0]
\prompt{Out}{outcolor}{ }{\boxspacing}
\begin{Verbatim}[commandchars=\\\{\}]
False
\end{Verbatim}
\end{tcolorbox}
        
    \begin{tcolorbox}[breakable, size=fbox, boxrule=1pt, pad at break*=1mm,colback=cellbackground, colframe=cellborder]
\prompt{In}{incolor}{ }{\boxspacing}
\begin{Verbatim}[commandchars=\\\{\}]
\PY{l+m+mi}{5186} \PY{o+ow}{in} \PY{n}{d}\PY{o}{.}\PY{n}{values}\PY{p}{(}\PY{p}{)}
\end{Verbatim}
\end{tcolorbox}

            \begin{tcolorbox}[breakable, size=fbox, boxrule=.5pt, pad at break*=1mm, opacityfill=0]
\prompt{Out}{outcolor}{ }{\boxspacing}
\begin{Verbatim}[commandchars=\\\{\}]
True
\end{Verbatim}
\end{tcolorbox}
        
    \begin{center}\rule{0.5\linewidth}{0.5pt}\end{center}

\hypertarget{activituxe9-23.3}{%
\subsubsection{Activité 23.3}\label{activituxe9-23.3}}

On considère le dictionnaire \texttt{fruits} suivant.

    \begin{tcolorbox}[breakable, size=fbox, boxrule=1pt, pad at break*=1mm,colback=cellbackground, colframe=cellborder]
\prompt{In}{incolor}{ }{\boxspacing}
\begin{Verbatim}[commandchars=\\\{\}]
\PY{n}{fruits} \PY{o}{=} \PY{p}{\PYZob{}}\PY{l+s+s1}{\PYZsq{}}\PY{l+s+s1}{poires}\PY{l+s+s1}{\PYZsq{}}\PY{p}{:} \PY{l+m+mi}{5}\PY{p}{,} \PY{l+s+s1}{\PYZsq{}}\PY{l+s+s1}{pommes}\PY{l+s+s1}{\PYZsq{}}\PY{p}{:} \PY{l+m+mi}{11}\PY{p}{,} \PY{l+s+s1}{\PYZsq{}}\PY{l+s+s1}{bananes}\PY{l+s+s1}{\PYZsq{}}\PY{p}{:} \PY{l+m+mi}{7}\PY{p}{,} \PY{l+s+s1}{\PYZsq{}}\PY{l+s+s1}{abricots}\PY{l+s+s1}{\PYZsq{}} \PY{p}{:} \PY{l+m+mi}{12}\PY{p}{\PYZcb{}}
\end{Verbatim}
\end{tcolorbox}

    \begin{enumerate}
\def\labelenumi{\arabic{enumi}.}
\tightlist
\item
  Affichez tous les fruits du dictionnaire.
\end{enumerate}

    \begin{tcolorbox}[breakable, size=fbox, boxrule=1pt, pad at break*=1mm,colback=cellbackground, colframe=cellborder]
\prompt{In}{incolor}{ }{\boxspacing}
\begin{Verbatim}[commandchars=\\\{\}]

\end{Verbatim}
\end{tcolorbox}

    \begin{enumerate}
\def\labelenumi{\arabic{enumi}.}
\setcounter{enumi}{1}
\tightlist
\item
  Affichez toutes les quantités du dictionnaire.
\end{enumerate}

    \begin{tcolorbox}[breakable, size=fbox, boxrule=1pt, pad at break*=1mm,colback=cellbackground, colframe=cellborder]
\prompt{In}{incolor}{ }{\boxspacing}
\begin{Verbatim}[commandchars=\\\{\}]

\end{Verbatim}
\end{tcolorbox}

    \begin{enumerate}
\def\labelenumi{\arabic{enumi}.}
\setcounter{enumi}{2}
\tightlist
\item
  Ecrivez un programme permettant d'obtenir l'affichage suivant.
\end{enumerate}

\begin{verbatim}
Il reste 5 poires
Il reste 11 pommes
Il reste 7 bananes
Il reste 12 abricots
\end{verbatim}

    \begin{tcolorbox}[breakable, size=fbox, boxrule=1pt, pad at break*=1mm,colback=cellbackground, colframe=cellborder]
\prompt{In}{incolor}{ }{\boxspacing}
\begin{Verbatim}[commandchars=\\\{\}]

\end{Verbatim}
\end{tcolorbox}

    Les dictionnaires : EXERCICES

    \hypertarget{exercice-23.1}{%
\section{Exercice 23.1 :}\label{exercice-23.1}}

On considère le dictionnaire suivant qui contient différents fruits
ainsi que leurs quantités.

    \begin{tcolorbox}[breakable, size=fbox, boxrule=1pt, pad at break*=1mm,colback=cellbackground, colframe=cellborder]
\prompt{In}{incolor}{ }{\boxspacing}
\begin{Verbatim}[commandchars=\\\{\}]
\PY{n}{fruits} \PY{o}{=} \PY{p}{\PYZob{}}\PY{l+s+s2}{\PYZdq{}}\PY{l+s+s2}{pommes}\PY{l+s+s2}{\PYZdq{}}\PY{p}{:} \PY{l+m+mi}{8}\PY{p}{,} \PY{l+s+s2}{\PYZdq{}}\PY{l+s+s2}{melons}\PY{l+s+s2}{\PYZdq{}}\PY{p}{:} \PY{l+m+mi}{3}\PY{p}{,} \PY{l+s+s2}{\PYZdq{}}\PY{l+s+s2}{poires}\PY{l+s+s2}{\PYZdq{}}\PY{p}{:} \PY{l+m+mi}{6}\PY{p}{\PYZcb{}}
\end{Verbatim}
\end{tcolorbox}

    \begin{enumerate}
\def\labelenumi{\arabic{enumi}.}
\tightlist
\item
  Quelle instruction permet d'accéder au nombre de melons ?
\end{enumerate}

    \begin{tcolorbox}[breakable, size=fbox, boxrule=1pt, pad at break*=1mm,colback=cellbackground, colframe=cellborder]
\prompt{In}{incolor}{ }{\boxspacing}
\begin{Verbatim}[commandchars=\\\{\}]

\end{Verbatim}
\end{tcolorbox}

    \begin{enumerate}
\def\labelenumi{\arabic{enumi}.}
\setcounter{enumi}{1}
\tightlist
\item
  On a acheté 16 clémentines et utilisé 4 pommes pour faire une tarte.
  Quelles instructions permettent de mettre à jour le dictionnaire ?
\end{enumerate}

    \begin{tcolorbox}[breakable, size=fbox, boxrule=1pt, pad at break*=1mm,colback=cellbackground, colframe=cellborder]
\prompt{In}{incolor}{ }{\boxspacing}
\begin{Verbatim}[commandchars=\\\{\}]

\end{Verbatim}
\end{tcolorbox}

    \hypertarget{exercice-23.2}{%
\section{Exercice 23.2 :}\label{exercice-23.2}}

Répondez aux questions suivantes \textbf{sans exécuter les scripts
proposés}. \emph{Vous les exécuterez pour vérifier vos réponses.} 1.
Qu'affiche le programme suivant ?

    \begin{tcolorbox}[breakable, size=fbox, boxrule=1pt, pad at break*=1mm,colback=cellbackground, colframe=cellborder]
\prompt{In}{incolor}{ }{\boxspacing}
\begin{Verbatim}[commandchars=\\\{\}]
\PY{n}{fruits} \PY{o}{=} \PY{p}{\PYZob{}}\PY{l+s+s1}{\PYZsq{}}\PY{l+s+s1}{pommes}\PY{l+s+s1}{\PYZsq{}}\PY{p}{:} \PY{l+m+mi}{4}\PY{p}{,} \PY{l+s+s1}{\PYZsq{}}\PY{l+s+s1}{melons}\PY{l+s+s1}{\PYZsq{}}\PY{p}{:} \PY{l+m+mi}{3}\PY{p}{,} \PY{l+s+s1}{\PYZsq{}}\PY{l+s+s1}{poires}\PY{l+s+s1}{\PYZsq{}}\PY{p}{:} \PY{l+m+mi}{6}\PY{p}{,} \PY{l+s+s1}{\PYZsq{}}\PY{l+s+s1}{clémentines}\PY{l+s+s1}{\PYZsq{}}\PY{p}{:} \PY{l+m+mi}{16}\PY{p}{\PYZcb{}}
\PY{k}{for} \PY{n}{c} \PY{o+ow}{in} \PY{n}{fruits}\PY{o}{.}\PY{n}{keys}\PY{p}{(}\PY{p}{)}\PY{p}{:}
    \PY{n+nb}{print}\PY{p}{(}\PY{n}{c}\PY{p}{)}
\end{Verbatim}
\end{tcolorbox}

    \begin{enumerate}
\def\labelenumi{\arabic{enumi}.}
\setcounter{enumi}{1}
\tightlist
\item
  Qu'affiche le programme suivant ?
\end{enumerate}

    \begin{tcolorbox}[breakable, size=fbox, boxrule=1pt, pad at break*=1mm,colback=cellbackground, colframe=cellborder]
\prompt{In}{incolor}{ }{\boxspacing}
\begin{Verbatim}[commandchars=\\\{\}]
\PY{n}{fruits} \PY{o}{=} \PY{p}{\PYZob{}}\PY{l+s+s1}{\PYZsq{}}\PY{l+s+s1}{pommes}\PY{l+s+s1}{\PYZsq{}}\PY{p}{:} \PY{l+m+mi}{4}\PY{p}{,} \PY{l+s+s1}{\PYZsq{}}\PY{l+s+s1}{melons}\PY{l+s+s1}{\PYZsq{}}\PY{p}{:} \PY{l+m+mi}{3}\PY{p}{,} \PY{l+s+s1}{\PYZsq{}}\PY{l+s+s1}{poires}\PY{l+s+s1}{\PYZsq{}}\PY{p}{:} \PY{l+m+mi}{6}\PY{p}{,} \PY{l+s+s1}{\PYZsq{}}\PY{l+s+s1}{clémentines}\PY{l+s+s1}{\PYZsq{}}\PY{p}{:} \PY{l+m+mi}{16}\PY{p}{\PYZcb{}}
\PY{k}{for} \PY{n}{cle}\PY{p}{,} \PY{n}{valeur} \PY{o+ow}{in} \PY{n}{fruits}\PY{o}{.}\PY{n}{items}\PY{p}{(}\PY{p}{)}\PY{p}{:}
    \PY{n+nb}{print}\PY{p}{(}\PY{n}{cle}\PY{p}{,} \PY{l+s+s2}{\PYZdq{}}\PY{l+s+s2}{\PYZhy{}\PYZgt{}}\PY{l+s+s2}{\PYZdq{}}\PY{p}{,} \PY{n}{valeur}\PY{p}{)}
\end{Verbatim}
\end{tcolorbox}

    \textbf{Réponse :}

    \begin{enumerate}
\def\labelenumi{\arabic{enumi}.}
\setcounter{enumi}{2}
\tightlist
\item
  Qu'affiche le programme suivant ?
\end{enumerate}

    \begin{tcolorbox}[breakable, size=fbox, boxrule=1pt, pad at break*=1mm,colback=cellbackground, colframe=cellborder]
\prompt{In}{incolor}{ }{\boxspacing}
\begin{Verbatim}[commandchars=\\\{\}]
\PY{n}{fruits} \PY{o}{=} \PY{p}{\PYZob{}}\PY{l+s+s1}{\PYZsq{}}\PY{l+s+s1}{pommes}\PY{l+s+s1}{\PYZsq{}}\PY{p}{:} \PY{l+m+mi}{4}\PY{p}{,} \PY{l+s+s1}{\PYZsq{}}\PY{l+s+s1}{melons}\PY{l+s+s1}{\PYZsq{}}\PY{p}{:} \PY{l+m+mi}{3}\PY{p}{,} \PY{l+s+s1}{\PYZsq{}}\PY{l+s+s1}{poires}\PY{l+s+s1}{\PYZsq{}}\PY{p}{:} \PY{l+m+mi}{6}\PY{p}{,} \PY{l+s+s1}{\PYZsq{}}\PY{l+s+s1}{clémentines}\PY{l+s+s1}{\PYZsq{}}\PY{p}{:} \PY{l+m+mi}{16}\PY{p}{\PYZcb{}}
\PY{k}{for} \PY{n}{v} \PY{o+ow}{in} \PY{n}{fruits}\PY{o}{.}\PY{n}{values}\PY{p}{(}\PY{p}{)}\PY{p}{:}
    \PY{n+nb}{print}\PY{p}{(}\PY{n}{v}\PY{p}{)}
\end{Verbatim}
\end{tcolorbox}

    \textbf{Réponse :}

    \hypertarget{exercice-23.3}{%
\section{Exercice 23.3}\label{exercice-23.3}}

On considère qu'il faut ajouter un fruit sur la liste des courses s'il
en reste 4 ou moins. 1. Ecrivez un programme qui affiche la liste des
courses en considérant le dictionnaire suivant.

    \begin{tcolorbox}[breakable, size=fbox, boxrule=1pt, pad at break*=1mm,colback=cellbackground, colframe=cellborder]
\prompt{In}{incolor}{ }{\boxspacing}
\begin{Verbatim}[commandchars=\\\{\}]
\PY{n}{fruits} \PY{o}{=} \PY{p}{\PYZob{}}\PY{l+s+s1}{\PYZsq{}}\PY{l+s+s1}{pommes}\PY{l+s+s1}{\PYZsq{}}\PY{p}{:} \PY{l+m+mi}{4}\PY{p}{,} \PY{l+s+s1}{\PYZsq{}}\PY{l+s+s1}{melons}\PY{l+s+s1}{\PYZsq{}}\PY{p}{:} \PY{l+m+mi}{3}\PY{p}{,} \PY{l+s+s1}{\PYZsq{}}\PY{l+s+s1}{poires}\PY{l+s+s1}{\PYZsq{}}\PY{p}{:} \PY{l+m+mi}{6}\PY{p}{,} \PY{l+s+s1}{\PYZsq{}}\PY{l+s+s1}{clémentines}\PY{l+s+s1}{\PYZsq{}}\PY{p}{:} \PY{l+m+mi}{16}\PY{p}{\PYZcb{}}
\PY{c+c1}{\PYZsh{} à compléter :}
\end{Verbatim}
\end{tcolorbox}

    \begin{enumerate}
\def\labelenumi{\arabic{enumi}.}
\setcounter{enumi}{1}
\tightlist
\item
  Ecrivez une fonction \texttt{liste\_courses(fruits)} qui prend en
  paramètre un dictionnaire \texttt{fruits} et qui renvoie un tableau
  avec les fruits de la liste de courses.
\end{enumerate}

    \begin{tcolorbox}[breakable, size=fbox, boxrule=1pt, pad at break*=1mm,colback=cellbackground, colframe=cellborder]
\prompt{In}{incolor}{ }{\boxspacing}
\begin{Verbatim}[commandchars=\\\{\}]

\end{Verbatim}
\end{tcolorbox}

    \hypertarget{exercice-23.4}{%
\section{Exercice 23.4 :}\label{exercice-23.4}}

    Voici deux dictionnaires :

    \begin{tcolorbox}[breakable, size=fbox, boxrule=1pt, pad at break*=1mm,colback=cellbackground, colframe=cellborder]
\prompt{In}{incolor}{ }{\boxspacing}
\begin{Verbatim}[commandchars=\\\{\}]
\PY{n}{athletes} \PY{o}{=} \PY{p}{\PYZob{}}\PY{l+s+s2}{\PYZdq{}}\PY{l+s+s2}{Mike}\PY{l+s+s2}{\PYZdq{}}\PY{p}{:} \PY{p}{(}\PY{l+m+mf}{1.75}\PY{p}{,} \PY{l+m+mi}{68}\PY{p}{)}\PY{p}{,} \PY{l+s+s2}{\PYZdq{}}\PY{l+s+s2}{John}\PY{l+s+s2}{\PYZdq{}}\PY{p}{:} \PY{p}{(}\PY{l+m+mf}{1.89}\PY{p}{,} \PY{l+m+mi}{93}\PY{p}{)}\PY{p}{,} \PY{l+s+s2}{\PYZdq{}}\PY{l+s+s2}{Kate}\PY{l+s+s2}{\PYZdq{}}\PY{p}{:} \PY{p}{(}\PY{l+m+mf}{1.67}\PY{p}{,} \PY{l+m+mi}{62}\PY{p}{)}\PY{p}{\PYZcb{}}
\PY{n}{sportifs} \PY{o}{=} \PY{p}{\PYZob{}}\PY{l+s+s2}{\PYZdq{}}\PY{l+s+s2}{Mike}\PY{l+s+s2}{\PYZdq{}}\PY{p}{:} \PY{p}{\PYZob{}}\PY{l+s+s2}{\PYZdq{}}\PY{l+s+s2}{taille}\PY{l+s+s2}{\PYZdq{}}\PY{p}{:} \PY{l+m+mf}{1.75}\PY{p}{,}\PY{l+s+s2}{\PYZdq{}}\PY{l+s+s2}{poids}\PY{l+s+s2}{\PYZdq{}}\PY{p}{:} \PY{l+m+mi}{68}\PY{p}{\PYZcb{}}\PY{p}{,} \PY{l+s+s2}{\PYZdq{}}\PY{l+s+s2}{John}\PY{l+s+s2}{\PYZdq{}}\PY{p}{:} \PY{p}{\PYZob{}}\PY{l+s+s2}{\PYZdq{}}\PY{l+s+s2}{taille}\PY{l+s+s2}{\PYZdq{}}\PY{p}{:} \PY{l+m+mf}{1.89}\PY{p}{,}\PY{l+s+s2}{\PYZdq{}}\PY{l+s+s2}{poids}\PY{l+s+s2}{\PYZdq{}}\PY{p}{:} \PY{l+m+mi}{93}\PY{p}{\PYZcb{}}\PY{p}{,} \PY{l+s+s2}{\PYZdq{}}\PY{l+s+s2}{Kate}\PY{l+s+s2}{\PYZdq{}}\PY{p}{:} \PY{p}{\PYZob{}}\PY{l+s+s2}{\PYZdq{}}\PY{l+s+s2}{taille}\PY{l+s+s2}{\PYZdq{}}\PY{p}{:} \PY{l+m+mf}{1.67}\PY{p}{,}\PY{l+s+s2}{\PYZdq{}}\PY{l+s+s2}{poids}\PY{l+s+s2}{\PYZdq{}}\PY{p}{:} \PY{l+m+mi}{62}\PY{p}{\PYZcb{}}\PY{p}{\PYZcb{}}
\end{Verbatim}
\end{tcolorbox}

    \begin{enumerate}
\def\labelenumi{\arabic{enumi}.}
\tightlist
\item
  De quel type sont les clés des deux dictionnaires \texttt{athletes} et
  \texttt{sportifs}? De quels types sont les valeurs de ces deux
  dictionnaires ?
\end{enumerate}

    \textbf{Réponse :}

    \begin{enumerate}
\def\labelenumi{\arabic{enumi}.}
\setcounter{enumi}{1}
\tightlist
\item
  Quelle instruction permet d'accéder à la taille de Kate dans le
  dictionnaire \texttt{athletes} ?
\end{enumerate}

    \begin{tcolorbox}[breakable, size=fbox, boxrule=1pt, pad at break*=1mm,colback=cellbackground, colframe=cellborder]
\prompt{In}{incolor}{ }{\boxspacing}
\begin{Verbatim}[commandchars=\\\{\}]

\end{Verbatim}
\end{tcolorbox}

    \begin{enumerate}
\def\labelenumi{\arabic{enumi}.}
\setcounter{enumi}{2}
\tightlist
\item
  Quelle instruction permet d'accéder à la taille de Kate dans le
  dictionnaire \texttt{sportifs} ?
\end{enumerate}

    \begin{tcolorbox}[breakable, size=fbox, boxrule=1pt, pad at break*=1mm,colback=cellbackground, colframe=cellborder]
\prompt{In}{incolor}{ }{\boxspacing}
\begin{Verbatim}[commandchars=\\\{\}]

\end{Verbatim}
\end{tcolorbox}

    \hypertarget{exercice-23.5}{%
\section{Exercice 23.5 :}\label{exercice-23.5}}

Le Scrabble est un jeu de société où l'on doit former des mots avec
tirage aléatoire de lettres, chaque lettre valant un certain nombre de
points. Le dictionnaire \texttt{scrabble} contient cette association
entre une lettre et son nombre de points.

    \begin{tcolorbox}[breakable, size=fbox, boxrule=1pt, pad at break*=1mm,colback=cellbackground, colframe=cellborder]
\prompt{In}{incolor}{ }{\boxspacing}
\begin{Verbatim}[commandchars=\\\{\}]
\PY{n}{scrabble} \PY{o}{=} \PY{p}{\PYZob{}}\PY{l+s+s1}{\PYZsq{}}\PY{l+s+s1}{A}\PY{l+s+s1}{\PYZsq{}}\PY{p}{:} \PY{l+m+mi}{1}\PY{p}{,} \PY{l+s+s1}{\PYZsq{}}\PY{l+s+s1}{B}\PY{l+s+s1}{\PYZsq{}}\PY{p}{:} \PY{l+m+mi}{3}\PY{p}{,} \PY{l+s+s1}{\PYZsq{}}\PY{l+s+s1}{C}\PY{l+s+s1}{\PYZsq{}}\PY{p}{:} \PY{l+m+mi}{3}\PY{p}{,} \PY{l+s+s1}{\PYZsq{}}\PY{l+s+s1}{D}\PY{l+s+s1}{\PYZsq{}}\PY{p}{:} \PY{l+m+mi}{2}\PY{p}{,} \PY{l+s+s1}{\PYZsq{}}\PY{l+s+s1}{E}\PY{l+s+s1}{\PYZsq{}}\PY{p}{:} \PY{l+m+mi}{1}\PY{p}{,} \PY{l+s+s1}{\PYZsq{}}\PY{l+s+s1}{F}\PY{l+s+s1}{\PYZsq{}}\PY{p}{:} \PY{l+m+mi}{4}\PY{p}{,} \PY{l+s+s1}{\PYZsq{}}\PY{l+s+s1}{G}\PY{l+s+s1}{\PYZsq{}}\PY{p}{:} \PY{l+m+mi}{2}\PY{p}{,} \PY{l+s+s1}{\PYZsq{}}\PY{l+s+s1}{H}\PY{l+s+s1}{\PYZsq{}}\PY{p}{:} \PY{l+m+mi}{4}\PY{p}{,} \PY{l+s+s1}{\PYZsq{}}\PY{l+s+s1}{I}\PY{l+s+s1}{\PYZsq{}}\PY{p}{:} \PY{l+m+mi}{1}\PY{p}{,} \PY{l+s+s1}{\PYZsq{}}\PY{l+s+s1}{J}\PY{l+s+s1}{\PYZsq{}}\PY{p}{:} \PY{l+m+mi}{8}\PY{p}{,} \PY{l+s+s1}{\PYZsq{}}\PY{l+s+s1}{K}\PY{l+s+s1}{\PYZsq{}}\PY{p}{:} \PY{l+m+mi}{10}\PY{p}{,} \PY{l+s+s1}{\PYZsq{}}\PY{l+s+s1}{L}\PY{l+s+s1}{\PYZsq{}}\PY{p}{:} \PY{l+m+mi}{1}\PY{p}{,} \PY{l+s+s1}{\PYZsq{}}\PY{l+s+s1}{M}\PY{l+s+s1}{\PYZsq{}}\PY{p}{:} \PY{l+m+mi}{2}\PY{p}{,} \PY{l+s+s1}{\PYZsq{}}\PY{l+s+s1}{N}\PY{l+s+s1}{\PYZsq{}}\PY{p}{:} \PY{l+m+mi}{1}\PY{p}{,} \PY{l+s+s1}{\PYZsq{}}\PY{l+s+s1}{O}\PY{l+s+s1}{\PYZsq{}}\PY{p}{:} \PY{l+m+mi}{1}\PY{p}{,} \PY{l+s+s1}{\PYZsq{}}\PY{l+s+s1}{P}\PY{l+s+s1}{\PYZsq{}}\PY{p}{:} \PY{l+m+mi}{3}\PY{p}{,} \PY{l+s+s1}{\PYZsq{}}\PY{l+s+s1}{Q}\PY{l+s+s1}{\PYZsq{}}\PY{p}{:} \PY{l+m+mi}{8}\PY{p}{,} \PY{l+s+s1}{\PYZsq{}}\PY{l+s+s1}{R}\PY{l+s+s1}{\PYZsq{}}\PY{p}{:} \PY{l+m+mi}{1}\PY{p}{,} \PY{l+s+s1}{\PYZsq{}}\PY{l+s+s1}{S}\PY{l+s+s1}{\PYZsq{}}\PY{p}{:} \PY{l+m+mi}{1}\PY{p}{,} \PY{l+s+s1}{\PYZsq{}}\PY{l+s+s1}{T}\PY{l+s+s1}{\PYZsq{}}\PY{p}{:} \PY{l+m+mi}{1}\PY{p}{,} \PY{l+s+s1}{\PYZsq{}}\PY{l+s+s1}{U}\PY{l+s+s1}{\PYZsq{}}\PY{p}{:} \PY{l+m+mi}{1}\PY{p}{,} \PY{l+s+s1}{\PYZsq{}}\PY{l+s+s1}{V}\PY{l+s+s1}{\PYZsq{}}\PY{p}{:} \PY{l+m+mi}{4}\PY{p}{,} \PY{l+s+s1}{\PYZsq{}}\PY{l+s+s1}{W}\PY{l+s+s1}{\PYZsq{}}\PY{p}{:} \PY{l+m+mi}{10}\PY{p}{,} \PY{l+s+s1}{\PYZsq{}}\PY{l+s+s1}{X}\PY{l+s+s1}{\PYZsq{}}\PY{p}{:} \PY{l+m+mi}{10}\PY{p}{,} \PY{l+s+s1}{\PYZsq{}}\PY{l+s+s1}{Y}\PY{l+s+s1}{\PYZsq{}}\PY{p}{:} \PY{l+m+mi}{10}\PY{p}{,} \PY{l+s+s1}{\PYZsq{}}\PY{l+s+s1}{Z}\PY{l+s+s1}{\PYZsq{}}\PY{p}{:} \PY{l+m+mi}{10}\PY{p}{\PYZcb{}}
\end{Verbatim}
\end{tcolorbox}

    Ecrivez une fonction \texttt{points(mot)} qui renvoie le nombre de
points au scrabble de \texttt{mot}, qui est une chaîne de caractères
majuscules.

\emph{Par exemple, le mot ``ARBRE'' doit rapporter 7 points, le mot
``XYLOPHONE'' doit rapporter 32 points}.

    \begin{tcolorbox}[breakable, size=fbox, boxrule=1pt, pad at break*=1mm,colback=cellbackground, colframe=cellborder]
\prompt{In}{incolor}{ }{\boxspacing}
\begin{Verbatim}[commandchars=\\\{\}]

\end{Verbatim}
\end{tcolorbox}

    \hypertarget{exercice-23.6}{%
\section{Exercice 23.6}\label{exercice-23.6}}

On considère la variable \texttt{personnages} suivante qui réunit
quelques informations sur des personnalités (les âges sont fictifs, vous
l'aurez compris).

    \begin{tcolorbox}[breakable, size=fbox, boxrule=1pt, pad at break*=1mm,colback=cellbackground, colframe=cellborder]
\prompt{In}{incolor}{ }{\boxspacing}
\begin{Verbatim}[commandchars=\\\{\}]
\PY{n}{personnages} \PY{o}{=} \PY{p}{[}\PY{p}{\PYZob{}}\PY{l+s+s1}{\PYZsq{}}\PY{l+s+s1}{nom}\PY{l+s+s1}{\PYZsq{}}\PY{p}{:} \PY{l+s+s1}{\PYZsq{}}\PY{l+s+s1}{Einstein}\PY{l+s+s1}{\PYZsq{}}\PY{p}{,} \PY{l+s+s1}{\PYZsq{}}\PY{l+s+s1}{prénom}\PY{l+s+s1}{\PYZsq{}}\PY{p}{:} \PY{l+s+s1}{\PYZsq{}}\PY{l+s+s1}{Albert}\PY{l+s+s1}{\PYZsq{}}\PY{p}{,} \PY{l+s+s1}{\PYZsq{}}\PY{l+s+s1}{âge}\PY{l+s+s1}{\PYZsq{}}\PY{p}{:} \PY{l+s+s1}{\PYZsq{}}\PY{l+s+s1}{35}\PY{l+s+s1}{\PYZsq{}}\PY{p}{,} \PY{l+s+s1}{\PYZsq{}}\PY{l+s+s1}{genre}\PY{l+s+s1}{\PYZsq{}}\PY{p}{:} \PY{l+s+s1}{\PYZsq{}}\PY{l+s+s1}{m}\PY{l+s+s1}{\PYZsq{}}\PY{p}{\PYZcb{}}\PY{p}{,}
           \PY{p}{\PYZob{}}\PY{l+s+s1}{\PYZsq{}}\PY{l+s+s1}{nom}\PY{l+s+s1}{\PYZsq{}}\PY{p}{:} \PY{l+s+s1}{\PYZsq{}}\PY{l+s+s1}{Hamilton}\PY{l+s+s1}{\PYZsq{}}\PY{p}{,} \PY{l+s+s1}{\PYZsq{}}\PY{l+s+s1}{prénom}\PY{l+s+s1}{\PYZsq{}}\PY{p}{:} \PY{l+s+s1}{\PYZsq{}}\PY{l+s+s1}{Margaret}\PY{l+s+s1}{\PYZsq{}}\PY{p}{,} \PY{l+s+s1}{\PYZsq{}}\PY{l+s+s1}{âge}\PY{l+s+s1}{\PYZsq{}}\PY{p}{:} \PY{l+s+s1}{\PYZsq{}}\PY{l+s+s1}{23}\PY{l+s+s1}{\PYZsq{}}\PY{p}{,} \PY{l+s+s1}{\PYZsq{}}\PY{l+s+s1}{genre}\PY{l+s+s1}{\PYZsq{}}\PY{p}{:} \PY{l+s+s1}{\PYZsq{}}\PY{l+s+s1}{f}\PY{l+s+s1}{\PYZsq{}}\PY{p}{\PYZcb{}}\PY{p}{,}
           \PY{p}{\PYZob{}}\PY{l+s+s1}{\PYZsq{}}\PY{l+s+s1}{nom}\PY{l+s+s1}{\PYZsq{}}\PY{p}{:} \PY{l+s+s1}{\PYZsq{}}\PY{l+s+s1}{Nelson}\PY{l+s+s1}{\PYZsq{}}\PY{p}{,} \PY{l+s+s1}{\PYZsq{}}\PY{l+s+s1}{prénom}\PY{l+s+s1}{\PYZsq{}}\PY{p}{:} \PY{l+s+s1}{\PYZsq{}}\PY{l+s+s1}{Ted}\PY{l+s+s1}{\PYZsq{}}\PY{p}{,} \PY{l+s+s1}{\PYZsq{}}\PY{l+s+s1}{âge}\PY{l+s+s1}{\PYZsq{}}\PY{p}{:} \PY{l+s+s1}{\PYZsq{}}\PY{l+s+s1}{64}\PY{l+s+s1}{\PYZsq{}}\PY{p}{,} \PY{l+s+s1}{\PYZsq{}}\PY{l+s+s1}{genre}\PY{l+s+s1}{\PYZsq{}}\PY{p}{:} \PY{l+s+s1}{\PYZsq{}}\PY{l+s+s1}{m}\PY{l+s+s1}{\PYZsq{}}\PY{p}{\PYZcb{}}\PY{p}{,}
           \PY{p}{\PYZob{}}\PY{l+s+s1}{\PYZsq{}}\PY{l+s+s1}{nom}\PY{l+s+s1}{\PYZsq{}}\PY{p}{:} \PY{l+s+s1}{\PYZsq{}}\PY{l+s+s1}{Curie}\PY{l+s+s1}{\PYZsq{}}\PY{p}{,} \PY{l+s+s1}{\PYZsq{}}\PY{l+s+s1}{prénom}\PY{l+s+s1}{\PYZsq{}}\PY{p}{:} \PY{l+s+s1}{\PYZsq{}}\PY{l+s+s1}{Marie}\PY{l+s+s1}{\PYZsq{}}\PY{p}{,} \PY{l+s+s1}{\PYZsq{}}\PY{l+s+s1}{âge}\PY{l+s+s1}{\PYZsq{}}\PY{p}{:} \PY{l+s+s1}{\PYZsq{}}\PY{l+s+s1}{41}\PY{l+s+s1}{\PYZsq{}}\PY{p}{,} \PY{l+s+s1}{\PYZsq{}}\PY{l+s+s1}{genre}\PY{l+s+s1}{\PYZsq{}}\PY{p}{:} \PY{l+s+s1}{\PYZsq{}}\PY{l+s+s1}{f}\PY{l+s+s1}{\PYZsq{}}\PY{p}{\PYZcb{}}\PY{p}{]}
\end{Verbatim}
\end{tcolorbox}

    \begin{enumerate}
\def\labelenumi{\arabic{enumi}.}
\tightlist
\item
  Quel est le type de la variable \texttt{personnages}? Quel est le type
  des éléments de \texttt{personnages} ?
\end{enumerate}

    \textbf{Réponse :}

    \begin{enumerate}
\def\labelenumi{\arabic{enumi}.}
\setcounter{enumi}{1}
\tightlist
\item
  Quelle instruction permet d'accéder au dictionnaire de Ted Nelson ?
\end{enumerate}

    \begin{tcolorbox}[breakable, size=fbox, boxrule=1pt, pad at break*=1mm,colback=cellbackground, colframe=cellborder]
\prompt{In}{incolor}{ }{\boxspacing}
\begin{Verbatim}[commandchars=\\\{\}]

\end{Verbatim}
\end{tcolorbox}

    \begin{enumerate}
\def\labelenumi{\arabic{enumi}.}
\setcounter{enumi}{2}
\tightlist
\item
  Quelle instruction permet d'accéder à l'âge de Ted Nelson ?
\end{enumerate}

    \begin{tcolorbox}[breakable, size=fbox, boxrule=1pt, pad at break*=1mm,colback=cellbackground, colframe=cellborder]
\prompt{In}{incolor}{ }{\boxspacing}
\begin{Verbatim}[commandchars=\\\{\}]

\end{Verbatim}
\end{tcolorbox}

    \begin{enumerate}
\def\labelenumi{\arabic{enumi}.}
\setcounter{enumi}{3}
\tightlist
\item
  Dans le programme suivant, quel est le type de la variable \texttt{p}
  à chaque tour de boucle ? Quel est le rôle de ce programme ?\\
\end{enumerate}

\begin{verbatim}
for p in personnages:
    if int(p['âge']) <= 40:
        print(p['nom'], p['prénom'])
\end{verbatim}

    \textbf{Réponse :}

    \begin{enumerate}
\def\labelenumi{\arabic{enumi}.}
\setcounter{enumi}{4}
\tightlist
\item
  Proposez un programme qui affiche uniquement les noms et prénoms des
  femmes du tableau \texttt{personnages}.
\end{enumerate}

    \begin{tcolorbox}[breakable, size=fbox, boxrule=1pt, pad at break*=1mm,colback=cellbackground, colframe=cellborder]
\prompt{In}{incolor}{ }{\boxspacing}
\begin{Verbatim}[commandchars=\\\{\}]

\end{Verbatim}
\end{tcolorbox}

    \begin{enumerate}
\def\labelenumi{\arabic{enumi}.}
\setcounter{enumi}{5}
\tightlist
\item
  Ecrivez une fonction \texttt{age\_moyen(personnages)} qui renvoie
  l'âge moyen des personnalités du tableau \texttt{personnages} entré en
  paramètre. \emph{On doit trouver 40,75 ans}.
\end{enumerate}

    \begin{tcolorbox}[breakable, size=fbox, boxrule=1pt, pad at break*=1mm,colback=cellbackground, colframe=cellborder]
\prompt{In}{incolor}{ }{\boxspacing}
\begin{Verbatim}[commandchars=\\\{\}]

\end{Verbatim}
\end{tcolorbox}


    % Add a bibliography block to the postdoc
    
    
    
\end{document}
